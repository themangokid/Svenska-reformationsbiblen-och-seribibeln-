\section*{Välkommen till Svenska Reformationsbibeln}

Detta är Svenska Reformationsbibelns hemsida. På denna hemsida finns det samlad information om Bibeln. Allt material på hemsidan är gratis.

Svenska Reformationsbibeln önskar väcka intresse för Bibeln. Ända sedan Sverige blev ett självständigt land har den svenska reformationsbibeln haft en stor betydelse för Sverige i dess utveckling till ett fritt och självständigt land och speciellt när det gäller språket. Vår fäderneärvda bibel är en dyrbar skatt som inte får gå förlorad!

\subsection*{Arbetets Framskridande}

Alla böcker i Bibeln är översatta. Nu pågår grammatikkontroll, kontroll av kursiv text och bibelhänvisningar samt layoutarbete. Nu är 27 böcker helt klara, dvs. Moseböckerna, Josua, Domarboken, Rut, 1 - 2 Samuelsboken, Esra, Nehemja, Ester, Höga Visan, Klagovisorna samt böckerna Hosea till Malaki. \textbf{Nuvarande prognos är att kunna skicka hela Bibelns text till tryckning mot slutet av 2025. Förbön och ekonomiskt stöd är mycket angeläget. Ny PDF-fil av GT är publicerad 19 juni 2025.}


\subsection*{Herren har bevarat sitt Ord}

Nyare översättningar såsom till exempel Bibel 2000 och Svenska Folkbibeln utgår ifrån en ny grundtext som kom på 1970-talet. Den nya grundtexten avviker från den gamla grundtexten på cirka 3 300 ställen i Nya Testamentet.

Vi som ger ut Svenska Reformationsbibeln förespråkar att vi håller fast vid den gamla grundtexten som använts av kristna under hela den kristna epoken. Vi tror att Herren har bevarat sitt ord.

Internationellt sett så försvarar även det engelska bibelsällskapet Trinitarian Bible Society, Textus Receptus, som ger ut King James Version i England. I USA finns bland annat de amerikanska organisationerna Way of Life och The Dean Burgon Society.

\subsection*{De kristna har varit ledda av den Helige Ande att välja ut rätt text}

Vi tror att de kristna var ledda av den Helige Ande till att välja ut de bibelböcker som skulle ingå i Nya Testamentet strax efter att de skrevs. Även om det finns invändningar bland vissa textkritiker mot detta påstående, t. ex. Kurt Aland, så är de flesta kristna överens om att de första kristna gjorde rätt val när de valde ut de 27 böcker som skulle ingå i Nya Testamentet och att inte acceptera de apokryfiska böckerna till Nya Testamentet. Vi anser att det är konsekvent och logiskt att tänka sig att de första kristna även var ledda av den Helige Ande till att också välja ut de rätta läsarterna från dem som var falska. I de flesta fall visste de varifrån handskriften kom, vem som hade skrivit den och om kopian var tillförlitlig eller en förfalskning. Det skedde tidiga försök att förfalska Skriften redan under andra århundradet. De första kristna såg till att korrekta och tillförlitliga handskrifter blev kopierade till nya exakta kopior. Framförallt inom den grekiska kyrkan skedde kopierandet med stor noggrannhet och vi har idag tusentals handskrifter som stämmer väl överens med varandra.

\subsection*{Den svenska reformationsbibeln följer den gamla grundtexten}

och önskar med Herrens hjälp se till att den får leva vidare i en nutida språkdräkt. Vid de få tillfällen när Karl XII:s kyrkobibel avviker från Textus Receptus så har vi med hjälp av grekiskkunnig expertis översatt Textus Receptus direkt. Det innebär att Reformationsbibeln är en grundtexttrogen översättning av Textus Receptus och vid de tillfällen som man kan välja olika tolkningsalternativ, så har vi som huvudregel valt den tolkning som finns i Karl XII:s kyrkobibel. Reformationsbibeln i Nya testamentet är därmed en tidsenlig översättning av den grekiska grundtexten Textus Receptus samtidigt som den också är en revidering av Karl XII:s kyrkobibel. Dessutom har den engelska översättningen King James Version utövat stort inflytande på arbetet i Nya Testamentet. Den som vill finna en svensk motsvarighet till King James Version skall också välja Reformationsbibeln, eftersom dessa översättningar troget följer samma grekiska grundtext.

\subsection*{Svenska Reformationsbibelns målsättning}

Det finns en hel del kristna idag som önskar att Karl XII:s Bibel skulle finnas tillgänglig i en språkdräkt som kan förstås av dagens människor. Den andra upplagan av Reformationsbibeln som nu är tryckt och klar är en direktöversättning av Textus Receptus samt även en revidering av Karls XII:s Bibel. Denna bibelutgåva har inte koppling till något specifikt samfund utan deltagarna i arbetet kommer från olika kristna sammanhang bland annat, pingstvänner, baptister och laestadianer. Att deltagarna kommer från olika kristna samfund har varit en garant för att inte något specifikt samfunds teologi har fått slå igenom i revideringsarbetet. Flera av deltagarna är utbildade i grekiska, vilket borgar för att denna revidering troget återger grundtextens betydelse. För mer information om vår målsättning, trosgrunder och vilka medarbetare som har medverkat finner du under menyknappen Om oss.

Den andra upplagan, trycktes i en upplaga 2016 som innehåller en utökad notapparat. Mer än 1 200 fotnoter kommer att upplysa om de skillnader som finns mellan de olika grekiska grundtexterna. Ytterligare en upplaga trycktes 2017 som innehåller cirka 800 fotnoter om vad som stod i Karl XII:s Bibel. I slutet finns en bilaga med alla viktiga grundtextskillnader.

\newpage